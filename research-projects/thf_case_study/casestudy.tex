\documentclass[11pt,]{article}
\usepackage[T1]{fontenc}
\usepackage{amssymb,amsmath}
\usepackage{txfonts}
\usepackage{microtype}
\usepackage{amssymb,amsmath}
\usepackage{graphicx}
\usepackage{subfigure} 
\usepackage{natbib}
\usepackage{paralist}
\usepackage{hyperref}	
\usepackage{url}
\usepackage{textcomp}
\urlstyle{same}
\usepackage{color}
\usepackage{fancyvrb}
\newcommand{\VerbBar}{|}
\newcommand{\VERB}{\Verb[commandchars=\\\{\}]}
\DefineVerbatimEnvironment{Highlighting}{Verbatim}{commandchars=\\\{\}}
% Add ',fontsize=\small' for more characters per line
\newenvironment{Shaded}{}{}
\newcommand{\KeywordTok}[1]{\textcolor[rgb]{0.00,0.44,0.13}{\textbf{{#1}}}}
\newcommand{\DataTypeTok}[1]{\textcolor[rgb]{0.56,0.13,0.00}{{#1}}}
\newcommand{\DecValTok}[1]{\textcolor[rgb]{0.25,0.63,0.44}{{#1}}}
\newcommand{\BaseNTok}[1]{\textcolor[rgb]{0.25,0.63,0.44}{{#1}}}
\newcommand{\FloatTok}[1]{\textcolor[rgb]{0.25,0.63,0.44}{{#1}}}
\newcommand{\CharTok}[1]{\textcolor[rgb]{0.25,0.44,0.63}{{#1}}}
\newcommand{\StringTok}[1]{\textcolor[rgb]{0.25,0.44,0.63}{{#1}}}
\newcommand{\CommentTok}[1]{\textcolor[rgb]{0.38,0.63,0.69}{\textit{{#1}}}}
\newcommand{\OtherTok}[1]{\textcolor[rgb]{0.00,0.44,0.13}{{#1}}}
\newcommand{\AlertTok}[1]{\textcolor[rgb]{1.00,0.00,0.00}{\textbf{{#1}}}}
\newcommand{\FunctionTok}[1]{\textcolor[rgb]{0.02,0.16,0.49}{{#1}}}
\newcommand{\RegionMarkerTok}[1]{{#1}}
\newcommand{\ErrorTok}[1]{\textcolor[rgb]{1.00,0.00,0.00}{\textbf{{#1}}}}
\newcommand{\NormalTok}[1]{{#1}}
\newcommand{\quotes}[1]{``#1''}

\hypersetup{breaklinks=true,
            pdfauthor={},
            pdftitle={},
            colorlinks=true,
            citecolor=blue,
            urlcolor=blue,
            linkcolor=magenta,
            pdfborder={0 0 0}}

\setlength{\parindent}{0pt}
\setlength{\parskip}{6pt plus 2pt minus 1pt}
\setlength{\emergencystretch}{3em}  % prevent overfull lines
\setcounter{secnumdepth}{1}

\usepackage[a4paper,body={170mm,250mm},top=25mm,left=25mm]{geometry}
\usepackage[sf,bf,small]{titlesec}
\usepackage{fancyhdr}

\pagestyle{fancy}
\lhead{\sffamily Sebastian Velasquez}
\rhead{\sffamily Exam No. B114657}
\cfoot{\sffamily \thepage}

\author{}
\date{}

\DeclareMathOperator{\softmax}{softmax}
\DeclareMathOperator{\sigmoid}{sigmoid}
\DeclareMathOperator{\sgn}{sgn}
\DeclareMathOperator{\relu}{relu}
\DeclareMathOperator{\lrelu}{lrelu}
\DeclareMathOperator{\elu}{elu}
\DeclareMathOperator{\selu}{selu}
\DeclareMathOperator{\maxout}{maxout}

\begin{document}

\begin{center}
\textsf{\textbf{\Large The Human Factor: Case Study}}

%\bigskip
%\textbf{Release date: Monday 6th November 2017}
%
%\textbf{Due date: 16:00 Tuesday 28th November 2017}
\end{center}

\section{Description of Problem}
\label{sec:problem}

\paragraph{}
The video streaming platform Netflix offers a wide range of alternatives for a broad audience. Its web user interface provides options to show details of a video like rating, category or reviews, however, it does not give an easy alternative to get the language information of a movie or series. In a scenario where a user wants to watch a video, she has to perform the task  \textbf{\quotes{Find out the language options of a video.}} The objective of this task is to make sure a video has the user\textquotesingle s preferred language options so she can watch it comfortably.

\paragraph{}
Currently, the task is performed by selecting a movie or series from the explorer, playing it, and clicking on the language options button in the bottom bar of the video player, as seen in Figure \ref{fig:fig1}. This sequence of actions requires that the user switches between two screens as seen in the flow graph in Figure \ref{fig:fig2}. Since there is no guarantee that a selected video has the preferred language options, the task could be repeated indefinitely.

\begin{figure*}[!htb]
\vskip 5mm
\begin{center}
\includegraphics[width=10cm]{image1}
\caption{Language options in video player}
\label{fig:fig1}
\end{center}
\vskip -5mm
\end{figure*}

\paragraph{}
In web environments, people expect to see the same page when navigating back in the site\citep{spool1999web}. Currently, the navigation between the explorer and the video player does not fulfill this expectation since the previous status of the explorer is lost when the user returns to it from the video player.

\begin{figure*}[tb]
\vskip 5mm
\begin{center}
\includegraphics[width=10cm]{image2}
\caption{Sequence of actions to perform task}
\label{fig:fig2}
\end{center}
\vskip -5mm
\end{figure*}

\section{Affected User Group}
\label{sec:users}
\paragraph{}
It is assumed that the users of the interface are familiarized with web interfaces and understand the type of interactions in these environments. Any user can be potentially affected by the problem, however, it is possible to classify them based on their language preferences:

\begin{itemize}
	\item \textbf{Group A:} People that only want to watch videos with audio in a given language.

	\item \textbf{Group B:} People that are willing to watch videos with subtitles, including hearing impairment people.
\end{itemize}

\paragraph{}
Due to the amount of effort required to perform the task, the current flow of actions can generate frustration in the users which could lead them to stop using the platform \citep{garrett2010elements}.

\section{Solution}
\label{sec:solution}

\paragraph{}
In order to improve the user experience and avoid the potential loss of users, the platform could implement one of the following alternatives:
\begin{itemize}
 \item \textbf{Language options for the profile:} Include an option in the profile administration section that allows the user to set language options (audio and subtitles) for videos. This way only movies and series that fulfill those criteria will be shown in the explorer. In this case, the task would be completely avoided. People from group A would find this option quite useful, however, if at some point they decide to watch videos in other languages they would have to reset their options and perform the task again.


\item \textbf{Language information in the explorer:} Display language information in the video card that is currently available in the explorer as seen in Figure \ref{fig:fig3}. At the moment, this card has a details section but no language information is shown.  In this case, the user still would perform the task, but the sequence of actions would be different as seen in the flow graph in Figure \ref{fig:fig4}. This new flow would reduce the time to perform the task since the transitions between screens would be avoided and the status of the explorer would be kept. Users from group B would find this option useful since they have a broader range of options.

\end{itemize}

\begin{figure*}[tb]
\vskip 5mm
\begin{center}
\includegraphics[width=10cm]{image4}
\caption{Item card of a video in the explorer}
\label{fig:fig3}
\end{center}
\vskip -5mm
\end{figure*}

\begin{figure*}[tb]
\vskip 5mm
\begin{center}
\includegraphics[width=10cm]{image3}
\caption{Proposed sequence of actions to perform task}
\label{fig:fig4}
\end{center}
\vskip -5mm
\end{figure*}


\bibliographystyle{plainnat}
\bibliography{casestudy-references}
\end{document}
