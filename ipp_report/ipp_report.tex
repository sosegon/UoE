%% Template for MLP Coursework 2 / 6 November 2017 

%% Based on  LaTeX template for ICML 2017 - example_paper.tex at 
%%  https://2017.icml.cc/Conferences/2017/StyleAuthorInstructions

\documentclass{article}

\usepackage[T1]{fontenc}
\usepackage{amssymb,amsmath}
\usepackage{txfonts}
\usepackage{microtype}

% For figures
\usepackage{graphicx}
\usepackage{subfigure} 

% For citations
\usepackage{natbib}

% For algorithms
\usepackage{algorithm}
\usepackage{algorithmic}

% the hyperref package is used to produce hyperlinks in the
% resulting PDF.  If this breaks your system, please commend out the
% following usepackage line and replace \usepackage{mlp2017} with
% \usepackage[nohyperref]{mlp2017} below.
\usepackage{hyperref}
\usepackage{url}
\urlstyle{same}

% Packages hyperref and algorithmic misbehave sometimes.  We can fix
% this with the following command.
\newcommand{\theHalgorithm}{\arabic{algorithm}}


% Set up MLP coursework style (based on ICML style)
\usepackage{mlp2017}
\mlptitlerunning{MLP Coursework 2 (\studentNumber)}
\bibliographystyle{icml2017}


\DeclareMathOperator{\softmax}{softmax}
\DeclareMathOperator{\sigmoid}{sigmoid}
\DeclareMathOperator{\sgn}{sgn}
\DeclareMathOperator{\relu}{relu}
\DeclareMathOperator{\lrelu}{lrelu}
\DeclareMathOperator{\elu}{elu}
\DeclareMathOperator{\selu}{selu}
\DeclareMathOperator{\maxout}{maxout}

%% You probably do not need to change anything above this comment

%% REPLACE this with your student number
\def\studentNumber{s1700260}

\begin{document} 

\twocolumn[
\mlptitle{IRR report\\ 
Gamifying Spaced Repetition Software}


\centerline{\studentNumber}

\vskip 7mm
]

\begin{abstract} 


\end{abstract} 

\section{Introduction}
Learning requires a considerable amount of mental effort. The amount of the mental effort expended in the learning process greatly depends on the perception of learners about the source of knowledge and the context of learning \citep{salomon1983differential}. It follows, that under conditions where people perceive the source of knowledge in a positive manner, the learning process requires less mental effort, therefore, it is easier, more pleasant, and provides better results \citep{salomon1984television}.

Traditional learning schemes have tried to include elements from new technologies with the objective of making the learning process more appealing for people. One of such approaches is known as serious games \citep{michael2005serious}. This alternative aims to take advantage of the pedagogical value of fun, competition and leisure to accomplish that objective. This technique has proved to provide benefits in the acquisition of new knowledge and skills \citep{graafland2012systematic}.

Even though the proven benefits of serious games, it is narrowed to specific fields and contexts. Moreover, the considerable number of resources necessitated to develop such type of solutions, makes it impractical in broader contexts. Under such circumstances, other alternatives have emerged to leverage the benefits of leisure and entertainment, but including capacities of flexibility and adaptation. One of such approaches is known as gamification.

The main objective of gamification is to increase the user engagement of products, services or processes by including game elements and principles to make them more appealing. The range of scenarios of usage of this technique is broader and includes organizational productivity \citep{zichermann2011gamification}, physical exercise \citep{hamari2015working} and learning \citep{hamari2016challenging}.

In the learning context, gamification has proved to be effective for different target audiences including primary school students \citep{boticki2015usage}, undergraduate students \citep{slish2015gamification}, and the general public \citep{disalvo2014khan}. The types of learning products, platforms and services that have implemented gamification are wide including mobile applications \citep{su2015mobile}, web sites \citep{gene2014gamification} and desktop software \citep{cheong2013quick}. These conditions ratify the flexible nature of gamification and its ability to be adapted to different environments.

Before the raise of technological techniques aimed to leverage the benefits of good perceptions of learning contexts, there were other alternatives aimed to facilitate the acquisition of new knowledge. One of these techniques was oriented to exploit the spacing effect phenomenon to increase the capacity of retention of specific contents. This technique is known as spaced repetition.

One of the foundations of spaced repetition is the acquisition of new content by recurrent revisions in a series of short learning sessions scattered over fixed or variable intervals of time. These conditions increase the capacity of retention compared to the acquirement of knowledge in a single massive learning session. Thus, the difficulty of remembering new facts, concepts or definitions is diminished and mental process like recognition and recall are boosted.

Spaced repetition has also taken advantage of new technologies. Thus, software products like Anki implements this technique to make it more accessible to a wider audience by providing web, desktop and mobile interfaces. 

\bibliography{ipp_report_references}

\end{document} 


% This document was modified from the file originally made available by
% Pat Langley and Andrea Danyluk for ICML-2K. This version was
% created by Lise Getoor and Tobias Scheffer, it was slightly modified  
% from the 2010 version by Thorsten Joachims & Johannes Fuernkranz, 
% slightly modified from the 2009 version by Kiri Wagstaff and 
% Sam Roweis's 2008 version, which is slightly modified from 
% Prasad Tadepalli's 2007 version which is a lightly 
% changed version of the previous year's version by Andrew Moore, 
% which was in turn edited from those of Kristian Kersting and 
% Codrina Lauth. Alex Smola contributed to the algorithmic style files.  
