%% Template for MLP Coursework 2 / 6 November 2017 

%% Based on  LaTeX template for ICML 2017 - example_paper.tex at 
%%  https://2017.icml.cc/Conferences/2017/StyleAuthorInstructions

\documentclass{article}

\usepackage[T1]{fontenc}
\usepackage{amssymb,amsmath}
\usepackage{txfonts}
\usepackage{microtype}

% For figures
\usepackage{graphicx}
\usepackage{subfigure} 

% For citations
\usepackage{natbib}

% For algorithms
\usepackage{algorithm}
\usepackage{algorithmic}

% the hyperref package is used to produce hyperlinks in the
% resulting PDF.  If this breaks your system, please commend out the
% following usepackage line and replace \usepackage{mlp2017} with
% \usepackage[nohyperref]{mlp2017} below.
\usepackage{hyperref}
\usepackage{url}
\urlstyle{same}

% Packages hyperref and algorithmic misbehave sometimes.  We can fix
% this with the following command.
\newcommand{\theHalgorithm}{\arabic{algorithm}}


% Set up MLP coursework style (based on ICML style)
\usepackage{mlp2017}
\mlptitlerunning{MLP Coursework 2 (\studentNumber)}
\bibliographystyle{icml2017}


\DeclareMathOperator{\softmax}{softmax}
\DeclareMathOperator{\sigmoid}{sigmoid}
\DeclareMathOperator{\sgn}{sgn}
\DeclareMathOperator{\relu}{relu}
\DeclareMathOperator{\lrelu}{lrelu}
\DeclareMathOperator{\elu}{elu}
\DeclareMathOperator{\selu}{selu}
\DeclareMathOperator{\maxout}{maxout}

%% You probably do not need to change anything above this comment

%% REPLACE this with your student number
\def\studentNumber{s1700260}

\begin{document} 

\twocolumn[
\mlptitle{IRR report\\ 
Gamifying Spaced Repetition Software}


\centerline{\studentNumber}

\vskip 7mm
]

\begin{abstract} 


\end{abstract} 

\section{Introduction}
Learning requires a considerable amount of mental effort. The amount of the mental effort expended in the learning process depends on the perception of the people about the source of learning and \citep{salomon1983differential}. It follows, that when people perceive the source of learning in a positive manner, the learning process needs less mental effort, therefore it is easier and more pleasant \citep{salomon1984television}.

Traditional learning schemes have tried to include elements from new technologies to make the learning process more appealing for people. One of such approaches is known as serious games \citep{michael2005serious}. This alternative aims to take advantage of the pedagogical value of fun, competition and leisure to accomplish that objective. This technique has proved to provide benefits in the acquisition of new knowledge and skills \citep{graafland2012systematic}.

Even though the proven benefits of serious games, it is narrowed to specific fields and contexts. Moreover, the considerable number of resources necessitated to develop such products, makes it impractical in broader contexts. Under such circumstances, there are other alternatives that also leverage the benefits of leisure and entertainment to make the learning process more appealing. One of such approaches is known as gamification.

The main objective of gamification is to include game elements and principles into existing products, services or process to make them more appealing. The range of scenario of usage of this technique is broad and includes organizational productivity \citep{zichermann2011gamification}, physical exercise \citep{hamari2015working} and learning \citep{hamari2016challenging}.

In the learning context, gamification has proved to be effective in different target audiences including undergraduate students \citep{slish2015gamification}

[REF]. Over the years, different kinds of techniques have been developed to facilitate the learning process. Depending on the kind of thing that is being learnt some techniques can provide better results [REF]. When it comes to learn facts, concepts or single stats, spaced repetition has been proved to increase the learning of the participants. This technique was first used by [REF] in XXXX, since then a multitude of variations and algorithm to set the spaces have been developed.

Due to a multitude of factors, learning new stuff can be cumbersome for some people. To increase the appealing to learn new stuff, different techniques have been developed. One of such techniques is the so called gamification. The objective of this technique is to increase the engagement of participants in product, services and other stuff by the usage of game principles.

Gamification has been applied in a broad range of scenarios including organizational productivity [REF], physical exercise [REF] and learnig[REF]. In all of this cases, the ultimate goal is to increase the user engagement. 


\bibliography{ipp_report_references}

\end{document} 


% This document was modified from the file originally made available by
% Pat Langley and Andrea Danyluk for ICML-2K. This version was
% created by Lise Getoor and Tobias Scheffer, it was slightly modified  
% from the 2010 version by Thorsten Joachims & Johannes Fuernkranz, 
% slightly modified from the 2009 version by Kiri Wagstaff and 
% Sam Roweis's 2008 version, which is slightly modified from 
% Prasad Tadepalli's 2007 version which is a lightly 
% changed version of the previous year's version by Andrew Moore, 
% which was in turn edited from those of Kristian Kersting and 
% Codrina Lauth. Alex Smola contributed to the algorithmic style files.  
